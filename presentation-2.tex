%% LyX 2.2.2 created this file.  For more info, see http://www.lyx.org/.
%% Do not edit unless you really know what you are doing.
\documentclass[english]{beamer}
\usepackage{mathptmx}
\usepackage[latin9]{inputenc}
\usepackage{amsmath}
\usepackage{amssymb}
\usepackage{graphicx}
\usepackage[european]{circuitikz}
\ctikzset{tripoles/mos style/arrows}

\makeatletter

%%%%%%%%%%%%%%%%%%%%%%%%%%%%%% LyX specific LaTeX commands.
%% Because html converters don't know tabularnewline
\providecommand{\tabularnewline}{\\}

%%%%%%%%%%%%%%%%%%%%%%%%%%%%%% Textclass specific LaTeX commands.
 % this default might be overridden by plain title style
 \newcommand\makebeamertitle{\frame{\maketitle}}%
 % (ERT) argument for the TOC
 \AtBeginDocument{%
   \let\origtableofcontents=\tableofcontents
   \def\tableofcontents{\@ifnextchar[{\origtableofcontents}{\gobbletableofcontents}}
   \def\gobbletableofcontents#1{\origtableofcontents}
 }

%%%%%%%%%%%%%%%%%%%%%%%%%%%%%% User specified LaTeX commands.
\usetheme{Warsaw}
% or ...

\setbeamercovered{transparent}
% or whatever (possibly just delete it)

\makeatother

\usepackage{babel}
\begin{document}

\title[Designing a SCaM]{Analog Design for a Single-Chip atto Mote (SCaM) Microcontroller
}

\author[Kevavi]{Kevin Chen, Avi Pandey, Kevin Zheng}

\institute{University of California, Berkeley}

\date{Presentation 1}

\makebeamertitle

%\pgfdeclareimage[height=0.5cm]{institution-logo}{logo.png}
%\logo{\pgfuseimage{institution-logo}}

\AtBeginSubsection[]{%
  \frame<beamer>{ 
    \frametitle{Outline}   
    \tableofcontents[currentsection,currentsubsection] 
  }
}

%\beamerdefaultoverlayspecification{<+->}
\begin{frame}{Outline}

\tableofcontents{}
\end{frame}

\section{Preliminary Results of Integration}

\subsection{Integration}
\begin{frame}{Semi-Ideal SCaM Overview}
\begin{itemize}
\item Made individual blocks with real components, but ideal op-amps
\item Tested each block separately, and met specs
\item Connected blocks together and dealt with loading issues
\item SCaM is alive!
\item Slowly adding completed, real op-amp blocks into the integration one by one to isolate problems caused by individual blocks \begin{itemize}
\item Strong Arm Latch
\item V$_{ref}$ regulator
\item PGA*
\item Digital regulator*
\end{itemize}
\end{itemize}
\end{frame}
%
\subsection{Error Results}

\begin{frame}{Ideal Full System LSB Error I}
\begin{itemize}
\item Selected analog input of 100 mV DC
\item PGA set to gain of 8
\item Tabulated worst case error in number of LSB
\end{itemize}
\begin{center}
Battery Voltage \\
\begin{tabular}{|c|c|c|c|}
\hline
& 1.6 V & 2.4 V & 3.2 V \tabularnewline \hline
70 $^\circ$C & & & \tabularnewline 
\hline
25 $^\circ$C & & & \tabularnewline
\hline
0 $^\circ$C & & & \tabularnewline 
\hline 
\end{tabular}
\end{center}
\end{frame}
%

\begin{frame}{Ideal Full System LSB Error II}
\begin{itemize}
\item Selected temperature sensor input
\item PGA set to gain of 1
\item Tabulated worst case error in number of LSB
\end{itemize}
\begin{center}
Battery Voltage \\
\begin{tabular}{|c|c|c|c|}
\hline
& 1.6 V & 2.4 V & 3.2 V \tabularnewline \hline
70 $^\circ$C & & & \tabularnewline 
\hline
25 $^\circ$C & & & \tabularnewline
\hline
0 $^\circ$C & & & \tabularnewline 
\hline 
\end{tabular}
\end{center}
\end{frame}
%

\begin{frame}{PGA LSB Error}
\begin{itemize}
\item Input of 200 mV
\item PGA set to gain of 4
\item Tabulated worst case error in number of LSB
\end{itemize}
\begin{center}
Battery Voltage \\
\begin{tabular}{|c|c|c|c|}
\hline
& 1.6 V & 2.4 V & 3.2 V \tabularnewline \hline
70 $^\circ$C & & & \tabularnewline 
\hline
25 $^\circ$C & & & \tabularnewline
\hline
0 $^\circ$C & & & \tabularnewline 
\hline 
\end{tabular}
\end{center}
\end{frame}
%

\section{Progress and Design Decisions}

\subsection{Ideal Blocks and Topologies}
\begin{frame}{BGT \& REG}
\begin{block}{Basic Performance}
\begin{itemize}
\item IS ALIVE WITH 1.2135-ISH VOLTS FOR BG/REGS, 1 VOLT FOR VREF
\end{itemize}
\end{block}
%
\begin{block}{Notable Design Decisions}
\begin{itemize}
\item Used PSR band gap reference - Li, Yao, Guo 2009
\item V$_{ref}$ LDO replaced with unity gain buffer
\end{itemize}
\end{block}
\end{frame}
%
\begin{frame}{MUX \& PGA}
\begin{block}{Basic Performance}
\begin{itemize}
\item MUX settles within 100 ns
\item PGA has $<$ 0.4\% error for all gains, and settles within 100ns
\end{itemize}
\end{block}
%
\begin{block}{Notable Design Decisions}
\begin{itemize}
\item Mostly minimum size transistors for switching
\item Used hvt devices for less leakage in transmission gates
\item Used nmos2v in parallel with $C_f$ across the op-amp
\item Used fairly large capacitors to prevent voltage drooping and charge injection
\end{itemize}
\end{block}
\end{frame}
%
\begin{frame}{ADC}
\begin{block}{Basic Performance}
\begin{itemize}
\item Accurate with an ideal supply
\item Uses a 10MHz digital clock
\end{itemize}
\end{block}
%
\begin{block}{Notable Design Decisions}
\begin{itemize}
\item Charge pump - Scott, Boser, Pister 2003
\end{itemize}
\end{block}
\end{frame}
%

\subsection{Clock Diagram}
\begin{frame}{Clock Diagram for PGA and ADC}
\end{frame}
%
\subsection{Real Op-amps (In Progress)}
\begin{frame}{BGT \& REG}
\begin{itemize}
\item Voltage Reference (finished):
\begin{itemize}
\item In unity gain configuration
\item Can handle anywhere from 8fF to $>$ 2pF loads
\end{itemize}
\item Digital Regulator ("finished"):
\begin{itemize}
\item Current version either oscillates very frequently or has a large variation from the intended band gap voltage
\item Planning to make a "slow" amplifier to avoid pulse-y behavior
\end{itemize}
\item Analog Regulator:
\begin{itemize}
\item "Slow"
\item Still need to think about a way to provide as little variation as possible
\end{itemize}
\item Band Gap Reference
\end{itemize}
\end{frame}
%
\begin{frame}{PGA}
\begin{itemize}
\item Two Stage PMOS Input Cascode:
\begin{itemize}
\item Having problems with leakage while holding across a clock cycle
\item Can drive to within 3mV of ground using second stage NMOS input
\end{itemize}
\item Topology:
\begin{itemize}
\item Extra switch added to discharge caps when V- cannot be driven to 0V fast enough
\end{itemize}
\end{itemize}
\end{frame}

\begin{frame}{ADC}
\begin{itemize}
\item Strong Arm Latch - Scott, Boser, Pister 2003:
\begin{itemize}
\item Used \textasciitilde RST = CLK
\end{itemize}
\end{itemize}
\end{frame}

\subsection{Responsibilities}
\begin{frame}{Responsibilities}

\begin{tabular}{|c|c|c|c|}
\hline 
 & ADC & PGA/MUX & BG/REG\tabularnewline
\hline 
\hline 
Topology & Pandey & Chen & Zheng\tabularnewline
\hline 
Amplifier & Zheng & Pandey & Chen\tabularnewline
\hline 
Verification & Chen & Zheng & Pandey\tabularnewline
\hline 
\end{tabular}
\end{frame}
%
\begin{frame}{Milestones}
\begin{itemize}
\item Blocks complete and implemented with ideal amplifiers (4/20)
\item System-level testbenches complete; first integration (4/21)
\item Ideal amplifiers replaced with real amplifiers (4/25)
\item Sanity check and verification (4/26)
\item Extras, final integration and submission (4/28)
\end{itemize}
\end{frame}
%
\begin{frame}{Scoring}
\begin{itemize}
\item 5\%: Meeting internal milestone dates
\item 55\%: Functional SCaM units meeting specs
\item 40\%: Functional SCaM system. 
\end{itemize}
\end{frame}

\end{document}
